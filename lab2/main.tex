\documentclass[12pt, a4paper]{article}
\usepackage{fullpage}
\usepackage[T1]{fontenc}
\usepackage{multicol}
\usepackage{amsmath}
\numberwithin{figure}{subsection}
\numberwithin{table}{subsection}
\usepackage{listings}
\lstset{
  language=C++,
  basicstyle=\ttfamily\small,
  keywordstyle=\color{blue}\bfseries,
  commentstyle=\color{gray},
  stringstyle=\color{green!50!black},
  breaklines=true,
  numbers=left,
  numberstyle=\tiny,
  frame=single,
  captionpos=b
}
\usepackage{array}
\usepackage{enumitem}
\usepackage{graphicx}
\usepackage{float}
\usepackage{adjustbox}
\usepackage{xcolor}
\usepackage{caption}
\usepackage{pgfplots}
\pgfplotsset{compat=1.18}
\usepackage{pgfplotstable}
\usepackage{hyperref}
\hypersetup{
    colorlinks=true,
    linkcolor=black,
    filecolor=black,      
    urlcolor=black,
    pdftitle={Laboratorium 2},
    pdfauthor={Marcel Duda, Jan Gawroński},
    pdfpagemode=FullScreen,
    }

\title{\huge{Algorytmy macierzowe} \break \Large{Laboratorium 2}}
\author{\large{Marcel Duda, Jan Gawroński}}
\date{17.11.2025}

\begin{document}
    \maketitle

    \section{Rekurencyjne odwracanie macierzy}
    \begin{equation}
    A = \begin{bmatrix}
    A_{11} & A_{12} \\
    A_{21} & A_{22}
    \end{bmatrix}
    \end{equation}
    \begin{equation}
    A^{-1} = \begin{bmatrix}
    A_{11}^{-1} + A_{11}^{-1} A_{12} S_{22}^{-1} A_{21} A_{11}^{-1} & -A_{11}^{-1} A_{12} S_{22}^{-1} \\
    -S_{22}^{-1} A_{21} A_{11}^{-1}  & S_{22}^{-1}
    \end{bmatrix}
    \end{equation}
    gdzie $S_{22} = A_{22} - A_{21} A_{11}^{-1} A_{12}$,
    jeśli $A$ jest macierzą $1 \times 1$, to $A^{-1} = \begin{bmatrix} \frac{1}{a_{11}} \end{bmatrix}$.

    \subsection{Implementacja}
    \begin{lstlisting}
Matrix inverse(const Matrix &A, std::unique_ptr<IMnozenie> &multImpl) {
    if (rows(A) == 1) {
        Matrix invA = zeroMatrix(1, 1);
        invA[0][0] = 1.0 / A[0][0];
        opCounterAdd({0, 0, 0, 1});
        return invA;
    }
    if (rows(A) % 2 == 0) {
        memCounterEnterCall(rows(A), cols(A), 3);
        int halfSize = rows(A) / 2;
        Matrix invA11 = inverse(subMatrix(A, 0, 0, halfSize, halfSize), multImpl);
        Matrix A12 = subMatrix(A, 0, halfSize, halfSize, halfSize);
        Matrix A21 = subMatrix(A, halfSize, 0, halfSize, halfSize);
        Matrix A22 = subMatrix(A, halfSize, halfSize, halfSize, halfSize);

        Matrix T1 = multImpl->multiply(invA11, A12);
        Matrix T2 = multImpl->multiply(A21, invA11);
        
        Matrix invS22 = inverse(A22 - multImpl->multiply(A21, T1), multImpl);

        Matrix T3 = multImpl->multiply(T1, invS22);

        Matrix B11 = invA11 + multImpl->multiply(T3, T2);
        Matrix B12 = negate(T3);
        Matrix B21 = negate(multImpl->multiply(invS22, T2));
        Matrix B22 = invS22;

        memCounterExitCall(rows(A), cols(A), 3);
        return combine(B11, B12, B21, B22);
    } else {
        Matrix A_padded = pad(A, rows(A) + 1, cols(A) + 1);
        A_padded[rows(A)][cols(A)] = 1.0;
        Matrix inv_padded = inverse(A_padded, multImpl);
        return trim(inv_padded, rows(A), cols(A));
    }
}
    \end{lstlisting}
    
    \subsection{Wykresy}
    \begin{figure}[H]
      \centering
      \begin{tikzpicture}
        \begin{axis}[
          width=\textwidth, height=0.42\textwidth,
          xlabel={Rozmiar macierzy (N)},
          ylabel={Czas [s]},
          grid=both,
          legend pos=north west,
        ]
          \addplot+[blue, mark=*, mark size=1, thick] table[x index=0,y index=1,col sep=space,comment chars=\#]{1.txt};
          \addplot+[red, mark=triangle*, mark size=1, thick]  table[x index=0,y index=1,col sep=space,comment chars=\#]{2.txt};
          \legend{Binet, Strassen}
        \end{axis}
      \end{tikzpicture}
      \caption{Czas działania (Binet vs Strassen)}
    \end{figure}

    % --- additions ---
    \begin{figure}[H]
      \centering
      \begin{tikzpicture}
        \begin{axis}[
          width=\textwidth, height=0.42\textwidth,
          xlabel={Rozmiar macierzy (N)},
          ylabel={Liczba dodawań},
          grid=both,
          legend pos=north west,
        ]
          \addplot+[blue, mark=*, mark size=1, thick] table[x index=0,y index=2,col sep=space,comment chars=\#]{1.txt};
          \addplot+[red, mark=triangle*, mark size=1, thick] table[x index=0,y index=2,col sep=space,comment chars=\#]{2.txt};
          \legend{Binet, Strassen}
        \end{axis}
      \end{tikzpicture}
      \caption{Porównanie liczby operacji dodawania}
    \end{figure}

    % --- subtractions ---
    \begin{figure}[H]
      \centering
      \begin{tikzpicture}
        \begin{axis}[
          width=\textwidth, height=0.42\textwidth,
          xlabel={Rozmiar macierzy (N)},
          ylabel={Liczba odejmowań},
          grid=both,
          legend pos=north west,
        ]
          \addplot+[blue, mark=*, mark size=1, thick] table[x index=0,y index=3,col sep=space,comment chars=\#]{1.txt};
          \addplot+[red, mark=triangle*, mark size=1, thick] table[x index=0,y index=3,col sep=space,comment chars=\#]{2.txt};
          \legend{Binet, Strassen}
        \end{axis}
      \end{tikzpicture}
      \caption{Porównanie liczby operacji odejmowania}
    \end{figure}

    % --- multiplications ---
    \begin{figure}[H]
      \centering
      \begin{tikzpicture}
        \begin{axis}[
          width=\textwidth, height=0.42\textwidth,
          xlabel={Rozmiar macierzy (N)},
          ylabel={Liczba mnożeń},
          grid=both,
          legend pos=north west,
        ]
          \addplot+[blue, mark=*, mark size=1, thick] table[x index=0,y index=4,col sep=space,comment chars=\#]{1.txt};
          \addplot+[red, mark=triangle*, mark size=1, thick] table[x index=0,y index=4,col sep=space,comment chars=\#]{2.txt};
          \legend{Binet, Strassen}
        \end{axis}
      \end{tikzpicture}
      \caption{Porównanie liczby operacji mnożenia}
    \end{figure}

    % --- divisions ---
    \begin{figure}[H]
      \centering
      \begin{tikzpicture}
        \begin{axis}[
          width=\textwidth, height=0.42\textwidth,
          xlabel={Rozmiar macierzy (N)},
          ylabel={Liczba operacji dzielenia},
          grid=both,
          legend pos=north west,
        ]
          \addplot+[blue, mark=*, mark size=1, thick] table[x index=0,y index=5,col sep=space,comment chars=\#]{1.txt};
          \addplot+[red, mark=triangle*, mark size=1, thick] table[x index=0,y index=5,col sep=space,comment chars=\#]{2.txt};
          \legend{Binet, Strassen}
        \end{axis}
      \end{tikzpicture}
      \caption{Porównanie liczby operacji dzielenia}
    \end{figure}

    % --- flops (sum) ---
    \begin{figure}[H]
      \centering
      \begin{tikzpicture}
        \begin{axis}[
          width=\textwidth, height=0.42\textwidth,
          xlabel={Rozmiar macierzy (N)},
          ylabel={Flopsy (suma operacji)},
          grid=both,
          legend pos=north west,
        ]
          \addplot+[blue, mark=*, mark size=1, thick] table[x index=0,y expr={\thisrowno{2}+\thisrowno{3}+\thisrowno{4}+\thisrowno{5}},col sep=space,comment chars=\#]{1.txt};
          \addplot+[red, mark=triangle*, mark size=1, thick] table[x index=0,y expr={\thisrowno{2}+\thisrowno{3}+\thisrowno{4}+\thisrowno{5}},col sep=space,comment chars=\#]{2.txt};
          \legend{Binet, Strassen}
        \end{axis}
      \end{tikzpicture}
      \caption{Porównanie liczby operacji zmiennoprzecinkowych (flops)}
    \end{figure}

    % --- pamięć (MB) ---
    \begin{figure}[H]
      \centering
      \begin{tikzpicture}
        \begin{axis}[
          width=\textwidth, height=0.42\textwidth,
          xlabel={Rozmiar macierzy (N)},
          ylabel={Pamięć szczytowa [MB]},
          grid=both,
          legend pos=north west,
        ]
          \addplot+[blue, mark=*, mark size=1, thick] table[x index=0,y expr={\thisrowno{6}/(1024*1024)},col sep=space,comment chars=\#]{1.txt};
          \addplot+[red, mark=triangle*, mark size=1, thick] table[x index=0,y expr={\thisrowno{6}/(1024*1024)},col sep=space,comment chars=\#]{2.txt};
          \legend{Binet, Strassen}
        \end{axis}
      \end{tikzpicture}
      \caption{Porównanie zużycia pamięci}
    \end{figure}

    % --- flops per second ---
    \begin{figure}[H]
      \centering
      \begin{tikzpicture}
        \begin{axis}[
          width=\textwidth, height=0.42\textwidth,
          xlabel={Rozmiar macierzy (N)},
          ylabel={Flopsy / s},
          grid=both,
          legend pos=north west,
        ]
          \addplot+[blue, mark=*, mark size=1, thick] table[x index=0,y expr={(\thisrowno{2}+\thisrowno{3}+\thisrowno{4}+\thisrowno{5})/max(1e-12,\thisrowno{1})},col sep=space,comment chars=\#]{1.txt};
          \addplot+[red, mark=triangle*, mark size=1, thick] table[x index=0,y expr={(\thisrowno{2}+\thisrowno{3}+\thisrowno{4}+\thisrowno{5})/max(1e-12,\thisrowno{1})},col sep=space,comment chars=\#]{2.txt};
          \legend{Binet, Strassen}
        \end{axis}
      \end{tikzpicture}
      \caption{Przepustowość (flops / s) porównanie}
    \end{figure}

    \section{Eliminacja Gaussa}
    Dane A, b:
    \begin{equation}
    A = \begin{bmatrix}
    A_{11} & A_{12} \\
    A_{21} & A_{22}
    \end{bmatrix}
    \end{equation}
    \begin{equation}
    b = \begin{bmatrix}
    b_1 \\
    b_2
    \end{bmatrix}
    \end{equation}
    \begin{equation}
    \begin{bmatrix}
    A_{11} & A_{12} \\
    A_{21} & A_{22}
    \end{bmatrix}
    \begin{bmatrix}
    x_1 \\
    x_2
    \end{bmatrix}
    =
    \begin{bmatrix}
    b_1 \\
    b_2
    \end{bmatrix}
    \end{equation}
    \begin{equation}
    A_{11} x_1 + A_{12} x_2 = b_1
    \end{equation}
    \begin{equation}
    A_{21} x_1 + A_{22} x_2 = b_2
    \end{equation}
    \begin{equation}
    A_{11} = L_{11} U_{11}
    \end{equation}
    \begin{equation}
    S = A_{22} - A_{21} U_{11}^{-1} L_{11}^{-1} A_{12} = L_S U_S
    \end{equation}
    \begin{equation}
    C = \begin{bmatrix}
    U_{11} & L_{11} A_{12} \\
    0 & U_S
    \end{bmatrix}
    \end{equation}
    \begin{equation}
    c = \begin{bmatrix}
    L_{11}^{-1} b_1 \\
    L_S^{-1} (b_2 - A_{21} U_{11}^{-1} L_{11}^{-1} b_1)
    \end{bmatrix}
    \end{equation}
    Zwróć C, c.

    \subsection{Implementacja}
    \begin{lstlisting}
std::pair<Matrix, Matrix> GaussElimination(const Matrix &A, const Matrix &b, std::unique_ptr<IMnozenie> &multImpl) {
    if (rows(A) == 1) {
        return {A, b};
    }

    if (rows(A) % 2 == 0) {
        memCounterEnterCall(rows(A), cols(A), 4);

        int halfSize = rows(A) / 2;

        Matrix A11 = subMatrix(A, 0, 0, halfSize, halfSize);
        Matrix A12 = subMatrix(A, 0, halfSize, halfSize, halfSize);
        Matrix A21 = subMatrix(A, halfSize, 0, halfSize, halfSize);
        Matrix A22 = subMatrix(A, halfSize, halfSize, halfSize, halfSize);

        Matrix b1 = subMatrix(b, 0, 0, halfSize, 1);
        Matrix b2 = subMatrix(b, halfSize, 0, halfSize, 1);

        auto [L11, U11] = LUfactorization(A11, multImpl);

        Matrix L11_inv = inverse(L11, multImpl);
        Matrix U11_inv = inverse(U11, multImpl);

        Matrix S1 = multImpl->multiply(A21, U11_inv);
        Matrix S2 = multImpl->multiply(L11_inv, A12);
        Matrix S3 = L11_inv * b1;

        auto [LS, US] = LUfactorization(A22 - multImpl->multiply(S1, S2), multImpl);

        Matrix LS_inv = inverse(LS, multImpl);

        Matrix c1 = S3;
        Matrix c2 = LS_inv * b2 - multImpl->multiply(LS_inv, S1) * S3;

        Matrix C11 = U11;
        Matrix C12 = multImpl->multiply(L11, A12);
        Matrix C21 = zeroMatrix(halfSize, halfSize);
        Matrix C22 = US;

        Matrix C = combine(C11, C12, 
                           C21, C22);
        Matrix c = combine(c1, {}, 
                           c2, {});

        memCounterExitCall(rows(A), cols(A), 4);
        return {C, c};
    } else {
        Matrix A_padded = pad(A, rows(A) + 1, rows(A) + 1);
        Matrix b_padded = pad(b, rows(b) + 1, 1);
        A_padded[rows(A)][rows(A)] = 1.0;
        b_padded[rows(b)][0] = 0.0;
        auto [C_padded, c_padded] = GaussElimination(A_padded, b_padded, multImpl);
        Matrix C = trim(C_padded, rows(A), rows(A));
        Matrix c = trim(c_padded, rows(b), 1);
        return {C, c};
    }
}
    \end{lstlisting}
    \subsection{Wykresy}
    \begin{figure}[H]
      \centering
      \begin{tikzpicture}
        \begin{axis}[
          width=\textwidth, height=0.42\textwidth,
          xlabel={Rozmiar macierzy (N)},
          ylabel={Czas [s]},
          grid=both,
          legend pos=north west,
        ]
          \addplot+[blue, mark=*, mark size=1, thick] table[x index=0,y index=1,col sep=space,comment chars=\#]{5.txt};
          \addplot+[red, mark=triangle*, mark size=1, thick]  table[x index=0,y index=1,col sep=space,comment chars=\#]{6.txt};
          \legend{Binet, Strassen}
        \end{axis}
      \end{tikzpicture}
      \caption{Czas działania (Binet vs Strassen)}
    \end{figure}

    % --- additions ---
    \begin{figure}[H]
      \centering
      \begin{tikzpicture}
        \begin{axis}[
          width=\textwidth, height=0.42\textwidth,
          xlabel={Rozmiar macierzy (N)},
          ylabel={Liczba dodawań},
          grid=both,
          legend pos=north west,
        ]
          \addplot+[blue, mark=*, mark size=1, thick] table[x index=0,y index=2,col sep=space,comment chars=\#]{5.txt};
          \addplot+[red, mark=triangle*, mark size=1, thick] table[x index=0,y index=2,col sep=space,comment chars=\#]{6.txt};
          \legend{Binet, Strassen}
        \end{axis}
      \end{tikzpicture}
      \caption{Porównanie liczby operacji dodawania}
    \end{figure}

    % --- subtractions ---
    \begin{figure}[H]
      \centering
      \begin{tikzpicture}
        \begin{axis}[
          width=\textwidth, height=0.42\textwidth,
          xlabel={Rozmiar macierzy (N)},
          ylabel={Liczba odejmowań},
          grid=both,
          legend pos=north west,
        ]
          \addplot+[blue, mark=*, mark size=1, thick] table[x index=0,y index=3,col sep=space,comment chars=\#]{5.txt};
          \addplot+[red, mark=triangle*, mark size=1, thick] table[x index=0,y index=3,col sep=space,comment chars=\#]{6.txt};
          \legend{Binet, Strassen}
        \end{axis}
      \end{tikzpicture}
      \caption{Porównanie liczby operacji odejmowania}
    \end{figure}

    % --- multiplications ---
    \begin{figure}[H]
      \centering
      \begin{tikzpicture}
        \begin{axis}[
          width=\textwidth, height=0.42\textwidth,
          xlabel={Rozmiar macierzy (N)},
          ylabel={Liczba mnożeń},
          grid=both,
          legend pos=north west,
        ]
          \addplot+[blue, mark=*, mark size=1, thick] table[x index=0,y index=4,col sep=space,comment chars=\#]{5.txt};
          \addplot+[red, mark=triangle*, mark size=1, thick] table[x index=0,y index=4,col sep=space,comment chars=\#]{6.txt};
          \legend{Binet, Strassen}
        \end{axis}
      \end{tikzpicture}
      \caption{Porównanie liczby operacji mnożenia}
    \end{figure}

    % --- divisions ---
    \begin{figure}[H]
      \centering
      \begin{tikzpicture}
        \begin{axis}[
          width=\textwidth, height=0.42\textwidth,
          xlabel={Rozmiar macierzy (N)},
          ylabel={Liczba operacji dzielenia},
          grid=both,
          legend pos=north west,
        ]
          \addplot+[blue, mark=*, mark size=1, thick] table[x index=0,y index=5,col sep=space,comment chars=\#]{5.txt};
          \addplot+[red, mark=triangle*, mark size=1, thick] table[x index=0,y index=5,col sep=space,comment chars=\#]{6.txt};
          \legend{Binet, Strassen}
        \end{axis}
      \end{tikzpicture}
      \caption{Porównanie liczby operacji dzielenia}
    \end{figure}

    % --- flops (sum) ---
    \begin{figure}[H]
      \centering
      \begin{tikzpicture}
        \begin{axis}[
          width=\textwidth, height=0.42\textwidth,
          xlabel={Rozmiar macierzy (N)},
          ylabel={Flopsy (suma operacji)},
          grid=both,
          legend pos=north west,
        ]
          \addplot+[blue, mark=*, mark size=1, thick] table[x index=0,y expr={\thisrowno{2}+\thisrowno{3}+\thisrowno{4}+\thisrowno{5}},col sep=space,comment chars=\#]{5.txt};
          \addplot+[red, mark=triangle*, mark size=1, thick] table[x index=0,y expr={\thisrowno{2}+\thisrowno{3}+\thisrowno{4}+\thisrowno{5}},col sep=space,comment chars=\#]{6.txt};
          \legend{Binet, Strassen}
        \end{axis}
      \end{tikzpicture}
      \caption{Porównanie liczby operacji zmiennoprzecinkowych (flops)}
    \end{figure}

    % --- pamięć (MB) ---
    \begin{figure}[H]
      \centering
      \begin{tikzpicture}
        \begin{axis}[
          width=\textwidth, height=0.42\textwidth,
          xlabel={Rozmiar macierzy (N)},
          ylabel={Pamięć szczytowa [MB]},
          grid=both,
          legend pos=north west,
        ]
          \addplot+[blue, mark=*, mark size=1, thick] table[x index=0,y expr={\thisrowno{6}/(1024*1024)},col sep=space,comment chars=\#]{5.txt};
          \addplot+[red, mark=triangle*, mark size=1, thick] table[x index=0,y expr={\thisrowno{6}/(1024*1024)},col sep=space,comment chars=\#]{6.txt};
          \legend{Binet, Strassen}
        \end{axis}
      \end{tikzpicture}
      \caption{Porównanie zużycia pamięci}
    \end{figure}

    % --- flops per second ---
    \begin{figure}[H]
      \centering
      \begin{tikzpicture}
        \begin{axis}[
          width=\textwidth, height=0.42\textwidth,
          xlabel={Rozmiar macierzy (N)},
          ylabel={Flopsy / s},
          grid=both,
          legend pos=north west,
        ]
          \addplot+[blue, mark=*, mark size=1, thick] table[x index=0,y expr={(\thisrowno{2}+\thisrowno{3}+\thisrowno{4}+\thisrowno{5})/max(1e-12,\thisrowno{1})},col sep=space,comment chars=\#]{5.txt};
          \addplot+[red, mark=triangle*, mark size=1, thick] table[x index=0,y expr={(\thisrowno{2}+\thisrowno{3}+\thisrowno{4}+\thisrowno{5})/max(1e-12,\thisrowno{1})},col sep=space,comment chars=\#]{6.txt};
          \legend{Binet, Strassen}
        \end{axis}
      \end{tikzpicture}
      \caption{Przepustowość (flops / s) porównanie}
    \end{figure}

    \section{LU faktoryzacja}
    \begin{equation}
    A = \begin{bmatrix}
    A_{11} & A_{12} \\
    A_{21} & A_{22}
    \end{bmatrix}
    \end{equation}
    \begin{equation}
    L_{11} U_{11} = LU(A_{11})
    \end{equation}
    \begin{equation}
    S = A_{22} - A_{21} U_{11}^{-1} L_{11}^{-1} A_{12} = L_S U_S
    \end{equation}
    \begin{equation}
    LU(A) = \begin{bmatrix}
    L_{11} & 0 \\
    A_{21} U_{11}^{-1} & L_S
    \end{bmatrix}
    \begin{bmatrix}
    U_{11} & L_{11}^{-1} A_{12} \\
    0 & U_S
    \end{bmatrix}
    \end{equation}
    
    Wyznacznik $A = LU$ to $\det(A) = \prod_{1}^{n} u_{nn}$.

    \subsection{Implementacja}
    \begin{lstlisting}
std::pair<Matrix, Matrix> LUfactorization(const Matrix &A, std::unique_ptr<IMnozenie> &multImpl) {
    if (rows(A) == 1) {
        Matrix L = identityMatrix(1);
        Matrix U = A;
        return {L, U};
    }
    
    if (rows(A) % 2 == 0) {
        memCounterEnterCall(rows(A), cols(A), 3);

        int halfSize = rows(A) / 2;

        Matrix A11 = subMatrix(A, 0, 0, halfSize, halfSize);
        Matrix A12 = subMatrix(A, 0, halfSize, halfSize, halfSize);
        Matrix A21 = subMatrix(A, halfSize, 0, halfSize, halfSize);
        Matrix A22 = subMatrix(A, halfSize, halfSize, halfSize, halfSize);

        auto [L11, U11] = LUfactorization(A11, multImpl);

        Matrix L11_inv = inverse(L11, multImpl);
        Matrix U11_inv = inverse(U11, multImpl);

        Matrix U12 = multImpl->multiply(L11_inv, A12);
        Matrix L21 = multImpl->multiply(A21, U11_inv);

        Matrix S = A22 - multImpl->multiply(L21, U12);

        auto [L22, U22] = LUfactorization(S, multImpl);

        Matrix L = combine(L11, zeroMatrix(halfSize, halfSize),
                           L21, L22);
        Matrix U = combine(U11, U12,
                           zeroMatrix(halfSize, halfSize), U22);

        memCounterExitCall(rows(A), cols(A), 3);
        return {L, U};
    } else {
        Matrix A_padded = pad(A, rows(A) + 1, rows(A) + 1);
        auto [L_padded, U_padded] = LUfactorization(A_padded, multImpl);
        Matrix L = trim(L_padded, rows(A), rows(A));
        Matrix U = trim(U_padded, rows(A), rows(A));
        return {L, U};
    }
}

double determinantLU(const Matrix &A, std::unique_ptr<IMnozenie> &multImpl) {
    auto [_, U] = LUfactorization(A, multImpl);
    double det = 1.0;
    for (int i = 0; i < rows(U); ++i) {
        det *= U[i][i];
    }
    return det;
}
    \end{lstlisting}

    \subsection{Wykresy}
        \begin{figure}[H]
      \centering
      \begin{tikzpicture}
        \begin{axis}[
          width=\textwidth, height=0.42\textwidth,
          xlabel={Rozmiar macierzy (N)},
          ylabel={Czas [s]},
          grid=both,
          legend pos=north west,
        ]
          \addplot+[blue, mark=*, mark size=1, thick] table[x index=0,y index=1,col sep=space,comment chars=\#]{3.txt};
          \addplot+[red, mark=triangle*, mark size=1, thick]  table[x index=0,y index=1,col sep=space,comment chars=\#]{4.txt};
          \legend{Binet, Strassen}
        \end{axis}
      \end{tikzpicture}
      \caption{Czas działania (Binet vs Strassen)}
    \end{figure}

    % --- additions ---
    \begin{figure}[H]
      \centering
      \begin{tikzpicture}
        \begin{axis}[
          width=\textwidth, height=0.42\textwidth,
          xlabel={Rozmiar macierzy (N)},
          ylabel={Liczba dodawań},
          grid=both,
          legend pos=north west,
        ]
          \addplot+[blue, mark=*, mark size=1, thick] table[x index=0,y index=2,col sep=space,comment chars=\#]{3.txt};
          \addplot+[red, mark=triangle*, mark size=1, thick] table[x index=0,y index=2,col sep=space,comment chars=\#]{4.txt};
          \legend{Binet, Strassen}
        \end{axis}
      \end{tikzpicture}
      \caption{Porównanie liczby operacji dodawania}
    \end{figure}

    % --- subtractions ---
    \begin{figure}[H]
      \centering
      \begin{tikzpicture}
        \begin{axis}[
          width=\textwidth, height=0.42\textwidth,
          xlabel={Rozmiar macierzy (N)},
          ylabel={Liczba odejmowań},
          grid=both,
          legend pos=north west,
        ]
          \addplot+[blue, mark=*, mark size=1, thick] table[x index=0,y index=3,col sep=space,comment chars=\#]{3.txt};
          \addplot+[red, mark=triangle*, mark size=1, thick] table[x index=0,y index=3,col sep=space,comment chars=\#]{4.txt};
          \legend{Binet, Strassen}
        \end{axis}
      \end{tikzpicture}
      \caption{Porównanie liczby operacji odejmowania}
    \end{figure}

    % --- multiplications ---
    \begin{figure}[H]
      \centering
      \begin{tikzpicture}
        \begin{axis}[
          width=\textwidth, height=0.42\textwidth,
          xlabel={Rozmiar macierzy (N)},
          ylabel={Liczba mnożeń},
          grid=both,
          legend pos=north west,
        ]
          \addplot+[blue, mark=*, mark size=1, thick] table[x index=0,y index=4,col sep=space,comment chars=\#]{3.txt};
          \addplot+[red, mark=triangle*, mark size=1, thick] table[x index=0,y index=4,col sep=space,comment chars=\#]{4.txt};
          \legend{Binet, Strassen}
        \end{axis}
      \end{tikzpicture}
      \caption{Porównanie liczby operacji mnożenia}
    \end{figure}

    % --- divisions ---
    \begin{figure}[H]
      \centering
      \begin{tikzpicture}
        \begin{axis}[
          width=\textwidth, height=0.42\textwidth,
          xlabel={Rozmiar macierzy (N)},
          ylabel={Liczba operacji dzielenia},
          grid=both,
          legend pos=north west,
        ]
          \addplot+[blue, mark=*, mark size=1, thick] table[x index=0,y index=5,col sep=space,comment chars=\#]{3.txt};
          \addplot+[red, mark=triangle*, mark size=1, thick] table[x index=0,y index=5,col sep=space,comment chars=\#]{4.txt};
          \legend{Binet, Strassen}
        \end{axis}
      \end{tikzpicture}
      \caption{Porównanie liczby operacji dzielenia}
    \end{figure}

    % --- flops (sum) ---
    \begin{figure}[H]
      \centering
      \begin{tikzpicture}
        \begin{axis}[
          width=\textwidth, height=0.42\textwidth,
          xlabel={Rozmiar macierzy (N)},
          ylabel={Flopsy (suma operacji)},
          grid=both,
          legend pos=north west,
        ]
          \addplot+[blue, mark=*, mark size=1, thick] table[x index=0,y expr={\thisrowno{2}+\thisrowno{3}+\thisrowno{4}+\thisrowno{5}},col sep=space,comment chars=\#]{3.txt};
          \addplot+[red, mark=triangle*, mark size=1, thick] table[x index=0,y expr={\thisrowno{2}+\thisrowno{3}+\thisrowno{4}+\thisrowno{5}},col sep=space,comment chars=\#]{4.txt};
          \legend{Binet, Strassen}
        \end{axis}
      \end{tikzpicture}
      \caption{Porównanie liczby operacji zmiennoprzecinkowych (flops)}
    \end{figure}

    % --- pamięć (MB) ---
    \begin{figure}[H]
      \centering
      \begin{tikzpicture}
        \begin{axis}[
          width=\textwidth, height=0.42\textwidth,
          xlabel={Rozmiar macierzy (N)},
          ylabel={Pamięć szczytowa [MB]},
          grid=both,
          legend pos=north west,
        ]
          \addplot+[blue, mark=*, mark size=1, thick] table[x index=0,y expr={\thisrowno{6}/(1024*1024)},col sep=space,comment chars=\#]{3.txt};
          \addplot+[red, mark=triangle*, mark size=1, thick] table[x index=0,y expr={\thisrowno{6}/(1024*1024)},col sep=space,comment chars=\#]{4.txt};
          \legend{Binet, Strassen}
        \end{axis}
      \end{tikzpicture}
      \caption{Porównanie zużycia pamięci}
    \end{figure}

    % --- flops per second ---
    \begin{figure}[H]
      \centering
      \begin{tikzpicture}
        \begin{axis}[
          width=\textwidth, height=0.42\textwidth,
          xlabel={Rozmiar macierzy (N)},
          ylabel={Flopsy / s},
          grid=both,
          legend pos=north west,
        ]
          \addplot+[blue, mark=*, mark size=1, thick] table[x index=0,y expr={(\thisrowno{2}+\thisrowno{3}+\thisrowno{4}+\thisrowno{5})/max(1e-12,\thisrowno{1})},col sep=space,comment chars=\#]{3.txt};
          \addplot+[red, mark=triangle*, mark size=1, thick] table[x index=0,y expr={(\thisrowno{2}+\thisrowno{3}+\thisrowno{4}+\thisrowno{5})/max(1e-12,\thisrowno{1})},col sep=space,comment chars=\#]{4.txt};
          \legend{Binet, Strassen}
        \end{axis}
      \end{tikzpicture}
      \caption{Przepustowość (flops / s) porównanie}
    \end{figure}

\end{document}
